%%%%%%%%%%%%%%%%%%%%%%%%%%%%%%%%%%%%%%%%%%%%%%%%%%%%%%%%%%%%
%%%%%%%%%%%%%%%%%%%%%%%%%%%%%%%%%%%%%%%%%%%%%%%%%%%%%%%%%%%%
%%%%%%%%%%%%%%%%%%%%%%%%%%%%%%%%%%%%%%%%%%%%%%%%%%%%%%%%%%%%
%%%%%%%%%%%%%%%%%%%%%%%%%%%%%%%%%%%%%%%%%%%%%%%%%%%%%%%%%%%%
%%%%%%%%%%%%%%%%%%%%%%%%%%%%%%%%%%%%%%%%%%%%%%%%%%%%%%%%%%%%
\documentclass[12pt]{article}
\usepackage{epsfig}
\usepackage{times}
\usepackage{xcolor}
\usepackage{color}
\usepackage{amsmath}
\usepackage{fancyhdr}
\renewcommand{\topfraction}{1.0}
\renewcommand{\bottomfraction}{1.0}
\renewcommand{\textfraction}{0.0}
\setlength {\textwidth}{6.6in}
\hoffset=-1.0in
\oddsidemargin=1.00in
\marginparsep=0.0in
\marginparwidth=0.0in                                                                               
\setlength {\textheight}{9.0in}
\voffset=-1.00in
\topmargin=1.0in
\headheight=0.0in
\headsep=0.00in
\footskip=0.50in                                         
\fancyfoot{}
\pagestyle{fancy}
\renewcommand{\headrulewidth}{0pt}
\fancyfoot[R]{31}
\fancyfoot[L]{Latex Example}
\begin{document}
\def\pos{\medskip\quad}
\def\subpos{\smallskip \qquad}
\newfont{\nice}{cmr12 scaled 1250}
\newfont{\name}{cmr12 scaled 1080}
\newfont{\swell}{cmbx12 scaled 800}
%%%%%%%%%%%%%%%%%%%%%%%%%%%%%%%%%%%%%%%%%%%%%%%%%%%%%%%%%%%%
%     DO NOT CHANGE ANYTHING ABOVE THIS LINE
%%%%%%%%%%%%%%%%%%%%%%%%%%%%%%%%%%%%%%%%%%%%%%%%%%%%%%%%%%%%
%     DO NOT CHANGE ANYTHING ABOVE THIS LINE
%%%%%%%%%%%%%%%%%%%%%%%%%%%%%%%%%%%%%%%%%%%%%%%%%%%%%%%%%%%%
%     DO NOT CHANGE ANYTHING ABOVE THIS LINE
%%%%%%%%%%%%%%%%%%%%%%%%%%%%%%%%%%%%%%%%%%%%%%%%%%%%%%%%%%%%

%%%%%%%%%%%%%%%%%%%%%%HW13 Begins Here%%%%%%%%%%%%%%%%%%%%%%

\begin{center}
{\large
PHYSICS 20323/60323: Fall 2023 - LaTeX Example
}\\
\end{center}

%%%%%%%%%%%%%%%%%%%%%%%%%%%%%%%%%%%%%%%%%%%%%%%%%%%%%%%%%%%%

\vskip0.05in
\noindent 1. {\bf The following questions refer to stars in the Table below.} \\
Note: There may be multiple answers.

%%%%%%%%%%%%%%%%%%%%%%%%%%%%%%%%%%%%%%%%%%%%%%%%%%%%%%%%%%%%

\vskip0.15in
\begin{tabular}{|l|c|r|r|r|r|}\hline
Name & Mass & Luminosity & Lifetime & Temperature & Radius \\\hline

$\eta$ Car. & 60.\(\textup{M}_\odot\) & $10^6$ \(\textup{L}_\odot\) & 8.0 $\times$ $10^5$ years &  & \\\hline

$\epsilon$ Eri. & 6.0\(\textup{M}_\odot\) & $10^3$ \(\textup{L}_\odot\) &  & 20,000 K & \\\hline

$\delta$ Scu. & 2.0\(\textup{M}_\odot\) & & 5.0 $\times$ $10^8$ years & & 2 \(\textup{R}_\odot\) \\\hline

$\beta$ Cyg. & 1.3\(\textup{M}_\odot\) & 3.5 \(\textup{L}_\odot\) & & & \\\hline

$\alpha$ Cen. & 1.0\(\textup{M}_\odot\) & & & & 1 \(\textup{R}_\odot\) \\\hline

$\gamma$ Del. & 0.7\(\textup{M}_\odot\) & & 4.5 $\times$ $10^{10}$ years & 5000 K & \\\hline
\end{tabular}

%%%%%%%%%%%%%%%%%%%%%%%%%%%%%%%%%%%%%%%%%%%%%%%%%%%%%%%%%%%%

\vskip0.3in
(a) (4 points) Which of these stars will produce a planetary nebula.

\vskip0.3in
(b) (4 points) Elements heavier than \textit{Carbon} will be produced in which stars.

%%%%%%%%%%%%%%%%%%%%%%%%%%%%%%%%%%%%%%%%%%%%%%%%%%%%%%%%%%%%

\vskip0.4in
\noindent 2. An electron is found to be in the spin state (in the \textit{z}-basis): $\chi = A\binom{3i}{4}$

\vskip0.3in
(a) (5 points) Determine the possible values of A such that the state is normalized.

\vskip0.3in
(b) (5 points) Find the expectation values of the operators \textcolor{red}{$S_x$}, \textcolor{violet}{$S_y$}, \textcolor{orange}{$S_z$} and $\vec{S}^2$

\vskip0.4in
The Matrix representations in the \textit{z}-basis for the components of electron spin operations are 

\vskip0.01in
given by:

\vskip0.17in
\begin{equation}
    \textcolor{red}{S_x = \frac{\textit{h}}{2}\binom{0 \\\\\ 1}{1 \\\\\ 0} \hspace{0.2cm}} \textcolor{violet}{; \hspace{1cm} S_y = \frac{\textit{h}}{2}\binom{0 \\ -i}{i \\\\\\\ 0} \hspace{0.2cm}} \textcolor{orange}{; \hspace{1cm} S_z = \frac{\textit{h}}{2}\binom{1 \\\\\\ 0}{0 \\ -1}}
\end{equation}

%%%%%%%%%%%%%%%%%%%%%%%%%%%%%%%%%%%%%%%%%%%%%%%%%%%%%%%%%%%%

\vskip0.3in
\noindent 3. The average electrostatic field in the earth's atmosphere in fair weather is approximately given:

\vskip0.1in
\begin{equation}
    \vec{E} = E_0 (Ae^{-\alpha z}+Be^{-\beta z}) \hat{z},
\end{equation}

\vskip0.1in
where A, B, $\alpha$, $\beta$ are positive constants and \textit{z} is the height above the (locally flat) earth surface.

\vskip0.2in
(a) (5 points) Find the average charge density in the atmosphere as a function of height

\vskip0.3in
(a) (5 points) Find the electric potential as a function height above the earth.

\end{document}
